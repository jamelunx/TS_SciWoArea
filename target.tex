%
% Copyright (C) 2001 by Holger Karl,
% karl@ft.ee.tu-berlin.de
%
% file: template.tex
%
% Time-stamp: Sat Oct 06 08:29:52 2001
%
% Template fuer die Ausarbeitungen zu TKN-Seminaren
%
\documentclass[12pt,twoside,doublepage]{article}

% Falls die Ausarbeitung in Deutsch erfolgt,
% die folgenden Kommentar-Zeichen '%' entfernen, andernfalls diese
 % Kommandos auskommentiert lassen:
% Languages:
% \usepackage[german]{babel}
% \usepackage[T1]{fontenc}
% \usepackage[latin1]{inputenc}
% \selectlanguage{german}
\usepackage{todonotes}
%%%%%%%%%%%%%%%%%%%%%%%%%%%
% Im restlichen Vorspann KEINE Aenderungen machen!
%%%%%%%%%%%%%%%%%%%%%%%%%%%
\usepackage{times}
\usepackage{url}
\usepackage[nolist]{acronym}

\usepackage{geometry}
\geometry{a4paper,body={5.8in,9in}}

% Graphics:
\usepackage{color}
\usepackage{graphicx}
% aller Bilder werden im Unterverzeichnis figures gesucht:
\graphicspath{{figures/}}

% Headers:
%\usepackage{fancyhdr}
%% \pagestyle{fancy}
%\pagestyle{fancy}
%\fancyhead{}
%\fancyhead[LE]{ \slshape \teilnehmer}
%\fancyhead[LO]{}
%\fancyhead[RE]{}
%\fancyhead[RO]{ \slshape \ausarbeitung}
%\fancyfoot[C]{}

% Projekt-Titel
\newcommand{\projecttitle}{\ \\ \ \\ \ \\Project group: Cooperative visual knowledge organization \\  (WS 2013/2014)}		


\title{
	\includegraphics[width=12cm]{figures/uni-logo}
	\ \\ \ \\
	\ \\ \ \\
	Target Specification: 
	 \ \\ \ \\ Scientific Working Area}
	
	
	
\author{\projecttitle}
\date{}      
                       % Activate to display a given date or no date
\usepackage{hyperref}
\usepackage{todonotes}



\begin{document}	
\maketitle
\ \\ \ \\
\ \\ \ \\

\begin{sloppypar}
	\centering \LARGE Members goes here	\ \\ \ \\  \begin{large} \date{\today} \end{large}	 
\end{sloppypar}

\thispagestyle{empty}
%Auf dieser Seite soll keine Seitennummer auftauchen
\clearpage
% eine neue Seite beginnen
%\setcounter{tocdepth}{2}
\tableofcontents
\thispagestyle{empty}
\clearpage


%%%%%%%%%%%%%%%%%%%%%%%%%%%%%%%%%%%%%%
% ab hier steht der eigentliche Text:

\pagenumbering{arabic}

\begin{acronym}[somethinglonger]
	\acro{MyAcro}{...}
\end{acronym}
	\clearpage 

\section[Definition of goals: Why is this software developed?]{Definition of goals: Why is this software developed?}


\section{How does the problem look like? Which structures and procedures exist?}


\section[What should the software do?]{What should the software do?}

\section{How should the software do this? Where should the software do this (environment)}

%%%%%%%%%%%%%%%%%%%%%%%%%%%%%%%%%%%%%%
\bibliography{bib}
\bibliographystyle{plain}

\end{document}


